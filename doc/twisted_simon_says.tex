%Document Class
\documentclass[12pt,letterpaper,oneside]{article}

%Packages
\usepackage{amsmath}
\usepackage[pdftex]{hyperref}
\usepackage[margin=1in]{geometry}
\usepackage{pdflscape}
\usepackage{setspace}
\usepackage{hyperref}
\usepackage{biblatex}
\bibliography{reference.bib}
\usepackage{listings}
\usepackage{MnSymbol}
\usepackage{tikz}
\usetikzlibrary{shapes,arrows}

%Settings
\setlength\parindent{0pt} %globally suppress indentation
\renewcommand{\rmdefault}{phv} % Arial
\renewcommand{\sfdefault}{phv} % Arial
\newcommand{\HRule}{\rule{\linewidth}{0.5mm}}
\graphicspath{{Images//}} %Path to pictures
\newcommand{\tab}{\hspace*{2em}} %Command for tab
\hypersetup{ %Sets up the hyperlinks in the TOC
colorlinks,
citecolor=black,
filecolor=black,
linkcolor=black,
urlcolor=black
}

\makeatletter
\lst@Key{numbers}{none}{%
\let\lst@PlaceNumber\@empty
    \lstKV@SwitchCases{#1}%
    {none&\\%
     left&\def\lst@PlaceNumber{\llap{\normalfont
                \lst@numberstyle{\thelstnumber}\kern\lst@numbersep}}\\%
     leftliteral&\def\lst@PlaceNumber{\llap{\normalfont
                \lst@numberstyle{\the\lst@lineno}\kern\lst@numbersep}}\\%
     right&\def\lst@PlaceNumber{\rlap{\normalfont
                \kern\linewidth \kern\lst@numbersep
                \lst@numberstyle{\thelstnumber}}}%
    }{\PackageError{Listings}{Numbers #1 unknown}\@ehc}}
\makeatother 

\makeatletter    
%\lst@UserCommand\lstlistlistingname{Code}
%\lst@UserCommand\lstlistingname{Code}

\def\lst@MProcessListing{%
    \lst@XPrintToken \lst@EOLUpdate \lsthk@InitVarsBOL
    \global\advance\lst@lineno\@ne
    \ifnum \lst@lineno>\lst@lastline
        \lst@ifdropinput \lst@LeaveMode \fi
        \ifx\lst@linerange\@empty
            \expandafter\expandafter\expandafter\lst@EndProcessListing
        \else
		\ifx\lst@OutputBox\@gobble\else
        \par\noindent \hbox{}%
    	\fi
          	{\bfseries\color{red} \ensuremath{\napprox}}
            \lst@interrange
            \lst@GetLineInterval
            \expandafter\expandafter\expandafter\lst@SkipToFirst
        \fi
    \else
        \expandafter\lst@BOLGobble
    \fi}
\makeatother
   
%Settings to properly display code in document
\lstloadlanguages{[GNU]C++, [x86masm]Assembler}
\lstset{%
tabsize=4,
numbers=leftliteral,
numberstyle=\tiny,
stepnumber=2,
numbersep=5pt,
basicstyle=\scriptsize\ttfamily,
breaklines=true,
escapeinside={\%*}{*)},
keywordstyle=\bfseries\color{green!40!black},
showstringspaces=false,
showspaces=false,
showtabs=false,
belowcaptionskip=6pt,
framexleftmargin=7mm,
frame=shadowbox,
rulesepcolor=\color{purple!60!black}
}
\lstset{prebreak=\raisebox{0ex}[0ex][0ex]
        {\ensuremath{\lcurvearrowdown}}}
\lstset{postbreak=\raisebox{0ex}[0ex][0ex]
        {\ensuremath{\rcurvearrowse\space}}}

\lstdefinestyle{custombash}{
  breaklines=true,
  xleftmargin=\parindent,
  language=bash,
  commentstyle=\itshape\color{blue},
  identifierstyle=\color{orange},
  stringstyle=\color{red},
}

\begin{document}

\begin{titlepage}
\vspace*{\fill}
\begin{center}
	
%Title
{\huge \bfseries "Twisted" Simon Says} \\[0.4cm]
Arduino Ignition Grant
\HRule \\[0.4cm]

% Authors/sponsors
\begin{minipage}{\textwidth}
\center
Zaza Soriano\\
Brian Taylor
\end{minipage}

\vspace*{\fill}

%Bottom of the page
{\large \today}
\end{center}
\end{titlepage}

\numberwithin{figure}{section} %Makes Figures numbered based on section
\numberwithin{table}{section}
\numberwithin{lstlisting}{section}
\pagenumbering{roman} %Page numbers roman before 1st section
\doublespace
\tableofcontents %Table of contentes
\newpage
%\listoffigures %Table of figures
%\newpage
\listoftables %Table of tables
\newpage
\lstlistoflistings %Table of listings
\newpage
\singlespace
\pagenumbering{arabic} %Page numbers normal from now on
			
\section{Goals}
%what is this? 

\section{References}
\url{http://www.instructables.com/id/Arduino-Simon-Says}\\
\url{http://www.instructables.com/id/Stickytape-Sensors/step5/Velostat}

\section{Hardware}
		\begin{itemize} \parskip0pt
			\item Arduino Board
			\item USB Cable
			\item Breadboard
			\item Wires
			\item 5 x Resistor 220$\Omega$
			\item 4 x LED
			\item 4 x Button
			\item Speaker
			\item Rotary Knob (Potentiometer)
			\item Light Sensor (Photocell)
			\item Motion Sensor (Tilt Switch)
			\item Temperature Sensor
			\item Conductive Fabric
			\item Conductive Thread
			\item Velostat
			\item Tape
			\item Scissors
		\end{itemize}

\section{Discussion}

	\subsection{Simon Says} \label{sec:simon}
Think fast... SIMON says, "Chase my flashing lights and sounds"! The challenge is to repeat the ever-increasing random signals that SIMON generates.\bigskip
\\SIMON is a computer-controlled game that consists of a base unit with 4 color lenses and a control panel.

	\subsection{Sensor Basics}	
			\subsubsection{Button}
			If already familiar with the button, please skip to Section \ref{sec:potentiometer}. 
			%Insert schematic w/explanation
			%Insert Code w/explanation

			\subsubsection{Potentiometer} \label{sec:potentiometer}
			If already familiar with the potentiometer, please skip to Section \ref{sec:photocell}. 
			%Insert schematic w/explanation
			%Insert Code w/explanation
			
			\subsubsection{Photocell} \label{sec:photocell}
			If already familiar with the photocell, please skip to Section \ref{sec:tilt}. 
			%Insert schematic w/explanation
			%Insert Code w/explanation
			
			\subsubsection{Tilt Switch} \label{sec:tilt}
			If already familiar with the tilt switch, please skip to Section \ref{sec:touch}. 
			%Insert schematic w/explanation
			%Insert Code w/explanation

			\subsubsection{Touch Switch} \label{sec:touch}
			If already familiar with the touch switch, please skip to Section \ref{sec:temp}. 
			%Insert schematic w/explanation
			%Insert Code w/explanation
				
			\subsubsection{Temperature Sensor} \label{sec:temp}
			If already familiar with the temperature sensor, please skip to Section \ref{sec:push}. 
			%Insert schematic w/explanation
			%Insert Code w/explanation
			
			\subsubsection{Custom Build Push Button} \label{sec:push}
			If already familiar with the custom built push button, please skip to Section \ref{sec:simon}. 
			%Insert schematic w/explanation
			%Insert Code w/explanation		
						
			\subsection{Sound Basics} \label{sec:sound}
			%insert schematic w/explanation
			%insert code w/explanation						

\section{Procedure}
		
		\subsection{"Not so twisted" Simon Says}
			\begin{enumerate}
				\item Wire the Arduino with 4 LED's, 4 push buttons, and 1 speaker (insert refs to images)
				\item Load the Twisted\_Simon\_Says.ino file onto the board.
			\end{enumerate}
								
		\subsection{"Twisted" Simon Says}
			\begin{enumerate}
				\item Replace one of the push buttons with an analog sensor
				\item 
			\end{enumerate}
		


\newpage
\appendix
\section{\\Arduino Code} \label{App:AppendixA}

	
\newpage		
\printbibliography

\end{document}
